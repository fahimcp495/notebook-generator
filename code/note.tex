\textbf{Binomial Coefficent}

- Factoring in: \( \binom{n}{k} = \frac{n}{k} \binom{n - 1}{k - 1} \)

- Sum over \( k \): \( \sum_{k = 0}^n \binom{n}{k} = 2^n \)

- Alternating sum: \( \sum_{k = 0}^n (-1)^k \binom{n}{k} = 0 \)

- Even and odd sum: \( \sum_{k = 0}^n \binom{n}{2k} = \sum_{k = 0}^n \binom{n}{2k + 1} 2^{n - 1} \)

- The Hockey Stick Identity

   - (Left to right) Sum over \( n \) and \( k \): \( \sum_{k = 0}^m \binom{n + k}{k} = \binom{n + m - 1}{m} \)

   - (Right to left) Sum over \( n \): \( \sum_{m = 0}^n \binom{m}{k} = \binom{n + 1}{k + 1} \)

- Sum of the squares: \( \sum_{k = 0}^n (\binom{n}{k})^2 = \binom{2n}{n} \)

- Weighted sum: \( \sum_{k = 1}^n k \binom{n}{k} = n2^{n - 1} \)

- Connection with the fibonacci numbers: \( \sum_{k = 0}\binom{n - k}{k} = F_{n + 1} \)

- Vandermonde's Identity: \( \sum_{i = 0}^k \binom{m}{i} \binom{n}{k - i} = \binom{m + n}{k} \)

\textbf{Fibonacci Number}

$$
k=A-B
$$
\begin{equation}
F_A F_B=F_{k+1} F_A^2+F_k F_A F_{A-1}
\end{equation}
\begin{equation}
\sum_{i=0}^n F_i^2=F_{n+1} F_n
\end{equation}
\begin{equation}
\sum_{i=0}^n F_i F_{i+1}=F_{n+1}^2-(-1)^n 
\end{equation}
\begin{equation}
    \sum_{i=0}^n F_i F_{i-1}=\sum_{i=0}^{n-1} F_i F_{i+1}
\end{equation}
\begin{equation}
G C D\left(F_m, F_n\right)=F_{G C D(m, n)}
\end{equation}
\begin{equation}
\sum_{0 \leq k \leq n}\left(\begin{array}{c}
n-k \\
k
\end{array}\right)=F i b_{n+1}
\end{equation}
\begin{equation}
\operatorname{gcd}\left(F_n, F_{n+1}\right)=\operatorname{gcd}\left(F_n, F_{n+2}\right)= \operatorname{gcd}\left(F_{n+1}, F_{n+2}\right)=1
\end{equation}

\textbf{Lucas Theorem}

\[ \binom{m}{n} \equiv \prod_{i = 0}^k \binom{m_i}{n_i} (\mod p) \]

- \( \binom{m}{n} \) is divisible by p if and only if at least one of the base-\(p\) digits of \( n \) is greater than the corresponding base-\( p \) digit of \( m \).

- The number of entries in the \( n \)th row of Pascal's triangle that are not divisible by \( p  = \prod_{i = 0}^k (n_i + 1) \)

- All entries in the \( (p^{k} - 1)th \) row are not divisble by \( p \).
- \( \binom{n}{m} \equiv \lfloor \frac{n}{p} \rfloor (\mod p) \)

\textbf{2nd Kaplansky's Lemma}

The number of ways of selecting \( k \) objects, no two consecutive,from \( n \) labelled objects arrayed in a circle is \( \frac{n}{k} \binom{n-k-1}{k-1} = \frac{n}{n - k} \binom{n-k}{k} \)

\textbf{Distinct Objects into Distinct Bins}

- $ n $ distinct objects into $ r $ distinct bins $ = r^n $

- Among $ n $ distinct objects, exactly $ k $ of them into r distincts bins $ = \binom{n}{k}r^k $

- $ n $ distinct objects into $ r $ distinct bins such that each bin contains at least one object $ = \sum_{i = 0}^{r} (-1)^i \binom{r}{i} (r - i)^n $

\textbf{Stirling Number 2nd Kind}

- Count the number of ways to partition a set of n labelled objects into k nonempty unlabelled subsets.
\[ S(n, k) = S(n-1, k-1) + k * S(n-1, k) \]
\[ S(0, 0) = 1, S(>0, 0) = 0, S(0, >0) = 0 \]

- Time Complexity: $ O(k\log n) $
\begin{verbatim}
ll get_sn2(int n, int k) {
  ll sn2 = 0;
  for (int i = 0; i <= k; ++i) {
    ll now = nCr(k, i) * powmod(k - i, n, mod) % mod;
    if (i&1) {
      now = now * (mod - 1) % mod;
    }
    sn2 = (sn2 + now) % mod;
  }
  sn2 = sn2 * ifact[k] % mod;
  return sn2;
}
\end{verbatim}
- Number of ways to color a 1\*n grid using k colors such that each color is used at least once = \( k! . sn2(n,k) \)

\textbf{Bell Numbers}

Counts the number of partitions of a set.
\begin{equation}
B_{n+1}=\sum_{k=0}^n\left(\frac{n}{k}\right) \cdot B_k
\end{equation}
$B_n=\sum_{k=0}^n S(n, k)$, where $S(n, k)$ is stirling number of second kind.

\textbf{Partition Number}

- Time Complexity: $ O(n\sqrt{n}) $
\begin{verbatim}
for (int i = 1; i <= n; ++i) {
  pent[2 * i - 1] = i * (3 * i - 1) / 2;
  pent[2 * i] = i * (3 * i + 1) / 2;
}
p[0] = 1;
for (int i = 1; i <= n; ++i) {
  p[i] = 0;
  for (int j = 1, k = 0; pent[j] <= i; ++j) {
    if (k < 2) p[i] = add(p[i], p[i - pent[j]]);
    else p[i] = sub(p[i], p[i - pent[j]]); ++k, k &= 3;
  }
}
\end{verbatim}
- The number of partitions of a positive integer \( n \) into exactly \( k \) parts equals the number of partitions of \( n \) whose largest part equals \( k \)

\[ p_k(n) = p_k(n - k) + p_{k - 1}(n - 1) \]

\textbf{Coloring}

- The number of labeled undirected graphs with \( n \) vertices, \( G_n = 2^{\binom{n}{2}} \)

- The number of labeled directed graphs with \( n \) vertices, \( G_n = 2^{n(n - 1)} \)

- The number of connected labeled undirected graphs with \( n \) vertices, \( C_n = 2^{\binom{n}{2}} - \frac{1}{n} \sum_{k = 1}^{n - 1} k \binom{n}{k} 2^{\binom{n-k}{2}}C_k = 2^{\binom{n}{2}} - \sum_{k = 1}^{n - 1} \binom{n - 1}{k - 1} 2^{\binom{n-k}{2}}C_k \)

- The number of k-connected labeled undirected graphs with \( n \) vertices, \( D[n][k] = \sum_{s = 1}^{n} \binom{n - 1}{s- 1}C_s D[n - s][k - 1] \)

- Cayley's formula: the number of trees on \( n \) labeled vertices = the number of spanning trees of a complete graph with \( n \) labeled vertices = \( n^{n - 2} \)

- Number of ways to color a graph using k color such that no two adjacent nodes have same color

  # Complete graph = \( k(k-1)(k-2)...(k-n+1) \)

  # Tree = \( k(k - 1)^{n - 1} \)

  # Cycle = \( (k - 1)^n + (-1)^n (k - 1) \)

- Number of trees with $ n $ labeled nodes: $ n^{n - 2} $

\textbf{Lucas Number}

Number of edge cover of a cycle graph $ C_n $ is $ L_n $

$ L(n) = L(n-1) + L(n-2); L(0)=2, L(1)=1 $

\textbf{Catalan Number}

\[ C_{n+1}= C_0C_n+C_1C_{n-1}+C_2C_{n-2}+... +C_nC_0 \]
\[ C_n = \binom{2n}{n} - \binom{2n}{n+1} \]
\[ C_n = \frac{1}{n + 1} \binom{2n}{n} \]

\textbf{Derangement}
\[ D_n = (n-1) (D_{n-1} + D_{n-2}) = nD_{n-1} + (-1)^n \]
\[ D_0 = 1, D_1 = 0 \]  
$ 1, 0, 1, 2, 9, 44, 265, ... $

\textbf{Ballot Theorem}

Suppose that in an election, candidate A receives
a votes and candidate B receives b votes, where a ≥ kb for some positive
integer k. Compute the number of ways the ballots can be ordered so that
A maintains more than k times as many votes as B throughout the counting
of the ballots.

The solution to the ballot problem is ((a − kb)/(a+b)) * C(a+b, a)

\textbf{Classical Problem}
$ F(n, k) = $ number of ways to color n objects using exactly $ k $ colors

Let $ G(n, k) $ be the number of ways to color n objects using no more than $ k $ colors.

Then, $ F(n, k) = G(n, k) - C(k, 1) * G(n, k-1) + C(k, 2) * G(n, k-2) - C(k, 3) * G(n, k-3) ... $

Determining G(n, k) :

Suppose, we are given a 1 * n grid. Any two adjacent cells can not have same color.
Then, G(n, k) = k * ((k-1)^(n-1))

If no such condition on adjacent cells.
Then, G(n, k) = k ^ n


\textbf{Generating Function}

$ 1/(1 - x) = 1 + x + x^2 + x^3 + ... $
$ 1/(1 - ax) = 1 + ax + (ax)^2 + (ax)^3 + ... $
$ 1/(1 - x)^2 = 1 + 2x + 3x^2 + 4x^3 + ... $
$ 1/(1 - x)^3 = C(2, 2) + C(3, 2)x + C(4, 2)x^2 + C(5, 2)x^3 + ... $
$ 1/(1 - ax)^(k + 1) = 1 + C(1 + k, k)(ax) + C(2 + k, k)(ax)^2 + C(3 + k, k)(ax)^3 + ... $
$ x(x + 1)(1 - x)^-3 = 1 + x + 4x^2 + 9x^3 + 16x^4 + 25x^5 + ... $
$ e^x = 1 + x + (x^2)/2! + (x^3)/3! + (x^4)/4! + ... $

\textbf{SUM}

\( 1^4+2^4+3^4+...+n^4=\frac{n(n+1)(2n+1)(3n2+3n+-1)}{30} \) \\
\( S_{(n, p)} = 1^p + 2^p + 3^p + 4^p + ... + n^ap \) \\
\( S(n, p) = \frac{1}{p + 1} [(n + 1)^{p + 1} - 1 - \sum_{i = 0}^{p - 1} \binom{p + 1}{i} S(n, i)] \) \\
\( 1.2 + 2.3 + 3.4 + ... = \frac{1}{3} n(n+1)(n+2) \) \\
\( \sum_{i=1}^n f_k(i) = \frac{1}{k+1} n(n+1)(n+2)...(n+k) = \frac{1}{k+1}\frac{(n+k)!}{(n-1)!} \) \\
\( \sum_{i=0}{n}ix^i = 1 + 2x^2 + 3x^3 + 4x^4 + 5x^5+ ... +nx^n = \frac{(x-(n+1)x^{n+1}+nx^{n+2})} {(x-1)^2} \) \\

\textbf{Probability}

\[ P(A|B) = \frac{P(A \cap B)}{P(B)} \]
\[ P(A|B) = \frac{P(B|A)P(A)}{P(B)} \]

\textbf{Matching Formula}

\textbf{Normal Graph} \\
MM + MEC = n (exculding vertex), IS + VC = G, MIS + MVC = G \\

\textbf{Bipartite Graph} \\
MIS = n - MBM, MVC = MBM, MEC = n - MBM \\

\textbf{Number Theory} \\
- HCN: 1e6(240), 1e9(1344), 1e12(6720), 1e14(17280), 1e15(26880), 1e16(41472) \\
- \( gcd(a, b, c, d, ...) = gcd(a, b - a, c - b, d - c, ...) \) \\
- \( gcd(a + k, b + k, c + k, d + k, ...) = gcd(a + k, b - a, c - b, d - c, ...) \) \\
- Primitive root exists iff \( n = 1, 2, 4, p^k, 2\times p^k \), where \( p \) is an odd prime. \\
- If primtive root exists, there are \( \phi(\phi(n)) \) primtive roots of \( n \). \\
- The numbers from \( 1 \) to \( n \) have in total \( O(n\log\log n) \) unique prime factors. \\
- \( x \equiv r_1 \mod m1 \) and \( x \equiv r_2 \mod m2 \) has a solution iff \( \gcd(m_1, m_2) | (r_1 - r_2) \) \\
Solution of \( x^2 \equiv a (\mod p) \):\\
- \( ca \equiv cb \pmod{m} \iff a \equiv b \pmod{ \frac{n}{\gcd(n, c)}} \) \\
- \( ax \equiv b \pmod{m} \) has a solution \( \iff \) \( \gcd(a, m) | b \) \\
- If \( ax \equiv b \pmod{m} \) has a solution, then it has \( gcd(a, m) \) solutions and they are separated by \( \frac{m}{\gcd(a, m)} \) \\
- \( ax \equiv 1 \pmod{m} \) has a solution or \( a \) is invertible \( \pmod{m} \) \( \iff \) \(\gcd(a, m) = 1 \) \\
- \( x^2 \equiv 1 \pmod{p} \) then \( x \equiv \pm 1 \pmod{p} \) \\
- There are \( \frac{p - 1}{2} \) has no solution. \\
- There are \( \frac{p - 1}{2} \) has exaclty two solutions. \\
- When \( p \% 4 = 3 \), \( x \equiv \pm a^{\frac{p + 1}{4}} \) \\
- When \( p \% 8 = 5 \), \( x \equiv a^{\frac{p + 3}{8}} \; or \; x \equiv 2^{\frac{p - 1}{4}} a^{\frac{p + 3}{8}} \)

\textbf{Totient}

- If \( p \) is a prime \( φ(p^k) = p^k - p^{k-1} \) \\
- If \( a \) & \( b \) are relatively prime, \( \phi(ab) = \phi(a)\phi(b) \) \\
- \( \phi(n) = n(1-\frac{1}{p_1})(1-\frac{1}{p_2})(1-\frac{1}{p_3})...(1-\frac{1}{p_k}) \) \\
- Sum of coprime to \( n = n * \frac{\phi(n)}{2} \) \\
- If \( n = 2^k, \phi(n) = 2^{k - 1} = \frac{n}{2} \) \\
- For \( a \) & \( b \), \( \phi(ab) = \phi(a)\phi(b)\frac{d}{\phi(d)} \) \\
- \( \phi (ip) = p \phi(i) \) whenever \( p \) is a prime and it divides \( i \) \\
- The number of \( a (1<= a <=N) \) such that \( gcd(a, N)=d \) is \( \phi(\frac{n}{d}) \) \\
- If \( n > 2 \) , \( \phi(n) \) is always even \\
- Sum of gcd, \( \sum_{i=1}^n gcd(i, n) = \sum_{d|n} d \phi(\frac{n}{d}) \) \\
- Sum of lcm, \( \sum_{i=1}{n}lcm(i, n) = \frac{n}{2}(\sum_{d|n}(d \phi(d))+1) \) \\
- \( \phi(1) = 1 \) and \( \phi(2) = 1 \) which two are only odd \( \phi \) \\
- \( \phi(3) = 2 \) and \( \phi(4) = 2 \) and \( \phi(6) = 2 \) which three are only prime \( \phi \) \\
- Find minimum n such that $ \frac{\phi(n)} {n} $ is  maximum- Multiple of small primes- $ 2 * 3 * 5 * 7 * 11 * 13 * ... $ \\

\textbf{Mobius}

\[ \sum_{i = 1}^n \sum_{j = 1}^n [gcd(i, j) = 1] = \sum_{k = 1}^n \mu(k) \lfloor \frac{n}{k} \rfloor^2 \]

\[ \sum_{i = 1}^n \sum_{j = 1}^n gcd(i, j) = \sum_{k = 1}^n k \sum_{l = 1}^{\lfloor \frac{n}{k} \rfloor} \mu(l) \lfloor {\frac{n}{kl}} \rfloor^2 \]

\[ \sum_{i = 1}^n \sum_{j = 1}^n gcd(i, j) = \sum_{k = 1}^n (\frac{\lfloor \frac{n}{k} \rfloor) (1 + \lfloor \frac{n}{k} \rfloor) }{2})^2 \sum_{d | k} \mu (d) kd \]

\textbf{Tree Hashing}

\( f(u) = sz[u] * \sum_{i = 0} f(v) * p^{i} \);    \( f(v) \) are sorted \)
\( f(child) = 1 \)

\textbf{Permutation} \\
- To maximize the sum of adjacent differences of a permuation, it is necessary and sufficient to place the smallest half numbers in odd position and the greatest half numbers in even position. Or, vice versa.

\textbf{String} \\
- If the sum of length of some strings is \( N \), there can be at most \( \sqrt{N} \) distinct length. \\
- A Text can have at most \( O(N \times \sqrt{N}) \) distinct substrings that match with given patterns where the sum of the length of the given patterns is \( N \). \\
- Period =  n \% (n - pi.back() == 0)? n - pi.back(): n \\
- The first \( (period) \) cyclic rotations of a string are distinct. Further cyclic rotations repeat the previous strings. \\
- \( S \) is a palindrome if and only if it's period is a palindrome. \\

\textbf{Bit} \\
- (a ⊕ b) and (a + b) has the same parity \\
- (a + b) = (a ⊕ b) + 2 (a & b) \\
- gcd(a, b) <= a - b <= xor(a, b) \\

\textbf{Convolution} \\
- Hamming Distance: Replace \( 0 \) with \( -1 \)
- SQRT Decomposition: Find block size, B = sqrt(8 * n)